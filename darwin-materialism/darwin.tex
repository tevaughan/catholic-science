
\documentclass[twocolumn]{article}

\usepackage{times}

\title{A Short Essay on Evolution, Materialism, and Classical Theism}
\author{Thomas E.~Vaughan}

\begin{document}

\maketitle

Charles Darwin's {\it%
   On the Origin of Species by Means of Natural Selection, or the Preservation
   of Favoured Races in the Struggle for Life%
},
published in 1859, had at least two significant consequences, one philosophical
and one scientific; the former consequence was extrinsic, and the latter,
intrinsic. The philosophical consequence was that biological evolution by
natural selection (henceforth ``evolution'') became associated with
materialism; this consequence was extrinsic because it came about merely by
virtue of the cultural circumstances in which the theory was introduced. The
scientific consequence, due to the intrinsic value of the theory, was that
evolution became a remarkably successful unifying principle for all of the
biological sciences. While the philosophical consequence seems to originate in
a reaction against a popular (though arguably inauthentic) view of God, the
scientific consequence led to a flowering of new research and technological
achievement. In any event, {\it On the Origin of Species\/} was a major
contribution to the scientific literature.

The idea of natural selection removed some of the apparent need for divine
interference in the realm of natural causes.\footnote{%
   Natural selection does not answer the question of how life began in the
   first place. Natural selection can in principle answer the question of how
   life, once begun in a single organism, could have given rise to the observed
   diversity. The lack of even a principle to explain the origin of the first
   living thing is a gap that remains open for divine interference with the
   chain of natural causes. However, to rely upon the idea of divine
   interference is not a properly scientific approach.
}
Although Darwin's version of evolutionary theory, which one might call
\emph{phyletic gradualism}, seems to be ruled out by the fossil record and was
replaced in 1972 by the \emph{punctuated equilibria} of Niles Eldredge and
Stephen Jay Gould, the central idea of natural selection as introduced by
Darwin remains.  If, in combination with changes in the environment, the random
variation of individuals within a species lead by selection of the fittest to
the emergence of new species, then the origin of a species need not require
God's interference with the ordinary functioning of the material world.

Materialism became associated with evolution at least in part because of the
Protestant English intellectual tradition of natural theology, perhaps best
summarized by William Paley in his {\it Natural Theology; or, Evidences of the
Existence and Attributes of the Deity}, published in 1802.  Paley pointed to
each of several amazing structures---like the eye---whose existence seems to
make sense only if, acting like a watchmaker, God interfered with natural
causes to introduce a being possessing both the amazing structure and the
ability to pass the structure on to progeny. One whose belief in God depended
in large part on such a view might be moved toward atheistic materialism if one
were impressed by evolutionary theory.  If God's special action be unnecessary
for the origin of every species, then perhaps God's action is not needed at
all, so the argument might go. The house built on the idea of God's necessary
interference with the chain of material causality was a house built on sand,
and Darwin's idea of natural selection was a rising tide that undermined the
foundation.

A problem with the Darwinian approach to materialism, however, is that not
every imaginable divine action is of the same kind, and not every kind of
divine action has the same importance. There is a qualitative difference
between God's \emph{creating} material things that interact with each other
according to their own natural causality and God's \emph{interfering} with
things that He creates. The elimination of the idea of divine action as the
proximate cause of natural phenomena is in fact a point of \emph{agreement}
between the Judeo-Christian tradition and the program of modern atheism.  Just
as the Jews opposed the nature gods of the cultures surrounding ancient Israel,
so too do modern atheists rightly resist the idea of the divine as a proximate
cause for ordinary happenings in nature.  However, such opposition is not aimed
against what is truly divine. The Jews believed in God without believing in
gods, and the modern Christian properly believes in God without reducing Him to
the level of Zeus, whether for the explanation of lightning or for the
explanation of the fossil record. Even if natural causes and random variation
produce new species, God is the creator of natural causes and random variation;
as the prophet Elija observed from the cave when the wind blew and the
earthquake shook the ground, the God of Abraham is not properly understood as
acting like a merely natural cause.

To demand---as Paley did and as those who promote Intelligent Design (ID) still
do today---to demand a god who will prove his existence by acting at the level
of natural causes is not to seek the Christian God.  Therefore, to deny the
demands of Paley or of ID is not to deny the Christian God.  Had England
resisted Protestantism, Darwin's theory might have been proposed without
substantial controversy, for the Catholic view of natural theology proves the
existence of God without demanding that God interfere like a pagan god in
ordinary events.

Natural selection does not deny God and therefore does not imply materialism.
Materialism became associated with evolution because of a failure to recognize
the transcendence of God. Materialism is mixed in with evolution just as, in a
carefully crafted conspiracy theory, an impossibility is mixed in with a
genuine and interesting possibility.  Materialism grew in popularity as the
conceptual benefits of evolution and other modern scientific theories were ever
more widely appreciated.

\end{document}

